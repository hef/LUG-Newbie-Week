\documentclass[hyperref={pdfpagelabels=false}]{beamer}
\usepackage{beamerthemesplit}
\usepackage{lmodern}
\title{Using the Shell}
\author{UIC Linux Users Group}
\date{\today}
\begin{document}
\frame{\titlepage}
\section{Using the Shell}
\frame
{
    \frametitle{Using the Shell}
    Using the Shell\\
    presented by Joel Luellwitz\\
    \begin{itemize}
    \item{Directory Navigation}
    \item{Permissions}
    \end{itemize}
    Newbies Week\\
    University of Illinois at Chicago\\
    Linux Users Group\\
    September 9th, 2010
}
\section{About the Presenter}
\frame
{
    \frametitle{About the Presenter}
    \begin{itemize}
%    \item{Alias: eviljoel (or ej)}
    \item{Name: Joel Luellwitz}
    \item{7 Years of Linux Experience}
    \item{UIC Alumni}
        \begin{itemize}
        \item{Management Information Systems Masters from the CBA}
        \end{itemize}
    \item{Currently work at Peapod.com}
    \end{itemize}
	
}
\subsection{Hold Your Comments}
\frame
{
Please hold your comments and contributions!
    \begin{itemize}
    \item{Questions are OK if you are confused or behind.}
    \item{May have time for questions at the end.}
    \end{itemize}
}
\section{Getting Started}
\subsection{Open a Shell}
\frame
{
    This is an interactive presentation.
    \begin{itemize}
    \item{Login}
        \begin{itemize}
        \item{Ask if you need help}
        \end{itemize}
    \item{ Click on Applications$\Rightarrow$Accessories$\Rightarrow$Terminal}
%    \item{ Type 'bash' (Enter) for a real shell.
    \end{itemize}
   
}
\subsection{A Shell, what is it?}
\frame
{
    \frametitle{A Shell, what is it?}
    \begin{itemize}
    \item{It interprets commands from the user.}
    \item{Passes control to other programs.}
    \item{bash and other shells}
    \end{itemize}
}
\section{Directory Navigation}
\subsection{List (ls)}
\frame
{
    \frametitle{List (ls)}
    Lists files and directories.
    \begin{itemize}
    \item{Directories are folders}
    \end{itemize}
    \begin{itemize}
    \item{ls}
    \item{ls -l}
    \item{ls -a}
    \item{ls -la}
    \end{itemize}
}
\frame
{
    \frametitle{List (ls) (continued)}
    \begin{itemize}
    \item{Wildcards}
    \item{more}
        \begin{itemize}    
        \item{ls -l | more}
        \end{itemize}
    \end{itemize}
}
\subsection{bash Tips}	
\frame
{
    \frametitle{Random bash Tips}
    \begin{itemize}
    \item{Tab Completion}
    \item{History}
    \end{itemize}
}
\subsection{Directory Commands}
\frame
{
    \frametitle{Directory Commands}
    \begin{itemize}
    \item{pwd}
    \item{mkdir}
    \item{rmdir}
    \end{itemize}
}
\subsection{Change Directory (cd)}
\frame
{
    \frametitle{Change Directory (cd)}
    Changes the current working directory.
    \begin{itemize}
    \item{cd <directory>}
    \item{cd /}
        \begin{itemize}
        \item{All locations start here}
        \item{No drives like on Windows or DOS}
        \end{itemize}
%    \item{cd \texttildelow (or just cd)}
        \begin{itemize}
        \item{Taks you to /home/<your username>}
        \end{itemize}
    \end{itemize}
}
\frame
{
    \frametitle{Change Directory (cd) - Relative Navigation}
    \begin{itemize}
    \item{cd ./}
    \item{cd ../}
    \item{cd ../../someotherplace/}
    \end{itemize}
}
\subsection{Mount Points}
\frame
{
    \frametitle{Mount Points}
    All physical volumes are part of the same hierarchy.
    \begin{itemize}
    \item{Mount points are found under /mnt and /media}
    \item{mount}
    \item{umount}
    \end{itemize}
}
\subsection{Moving, Copying and Deleting}
\frame
{
    \frametitle{Moving, Copying and Deleting}
    \begin{itemize}
    \item{touch}
        \begin{itemize}
        \item{Use it to create empty files}
        \end{itemize}
    \item{mv -i}
	\begin{itemize}
        \item{Rename and Move}
        \end{itemize}
    \item{cp -i}
    \item{rm -i}
	\begin{itemize}
        \item{rm -rfI}
        \end{itemize}
    \end{itemize}
}
\section{Permissions}
\subsection{Example Shell Script}
\frame
{	
    \frametitle{Example Shell Script}
%    echo {\textbackslash}#{\textbackslash}!/bin/sh > temp.sh\\
%    echo echo {\textbackslash}"Hello world{\textbackslash}!{\textbackslash}{\textbackslash}n{\textbackslash}" >> temp.sh
}
\frame
{
    \frametitle{Example Shell Script (continued)}
    \begin{itemize}
    \item{cat temp.sh}
    \item{echo}
%    \item{#!}
    \item{file redirection}
    \end{itemize}
}
\subsection{File Permissions}
\frame
{
    \frametitle{File Permissions}
    \begin{itemize}
    \item{chmod <permission> file(s)}
        \begin{itemize}
        \item{Read u+r (r--)}
        \item{Write u+w (-w-)}
        \item{Execute u+x (--x)}
            \begin{itemize}
            \item{chmod u+x temp.sh}
            \item{./temp.sh}
            \end{itemize}
        \end{itemize}
    \item{ls -l}
    \end{itemize}
}
\subsection{Directory Permissions}
\frame
{
    \frametitle{Directory Permissions}
    \begin{itemize}
    \item{chmod <permission> directory(ies)}
        \begin{itemize}
        \item{Read u+r (r--)}
        \item{Write (-w-)}
        \item{Traverse (--x)}
        \end{itemize}
    \item{ls -l}
        \begin{itemize}
        \item{d---------}
        \end{itemize}
    \end{itemize}
}
\subsection{User, Group, Other Permissions}
\frame
{
    \frametitle{User, Group, Other Permissions}
    \begin{itemize}
    \item{chmod u+rwx file(s) (-rwx------)}
    \item{chmod g+rwx file(s) (----rwx---)}
    \item{chmod o+rwx file(s) (-------rwx)}
    \end{itemize}
    \begin{itemize}
    \item{chmod u-rw file(s)}
    \item{chmod a+rx file(s)}
    \item{chmod 755 file(s) (-rwxr-wr-w)}
    \end{itemize}
}
\subsection{Complex Permissions}
\frame
{
    \frametitle{User, Group, Other Permissions}
    \begin{itemize}
    \item{-rwsr-Sr-t}
    \item{Layered directories to support complex permissions.}
    \item{Access Control Lists}
        \begin{itemize}
        \item{Security Enhanced Linux}
        \end{itemize}
    \end{itemize}
}
\subsection{Symlinks}
\frame
{
    \frametitle{Symlinks}
    \begin{itemize}
    \item{Can be used for security.}
    \item{Hard and Soft Links}
    \item{ln -s <realfile> <symlink>}
        \begin{itemize}
        \item{Like a Windows Shortcut}
        \item{More Transparent}
        \end{itemize}
    \end{itemize}
}
\section{Other Important Commands and Applications}
\subsection{Environment Variables}
\frame
{
    \frametitle{Environment Variables}
    \begin{itemize}
    \item{echo \$TERM}
    \item{export TERM="xterm"}
    \item{\$SHELL \$DISPLAY \$HOSTNAME \$JAVA \$PATH}
    \end{itemize}
}
\subsection{Text Editors}
\frame
{	
    \frametitle{Text Editors}
    \begin{itemize}
    \item{Newbies use pico or nano}
    \item{Real Linux users use vim}
        \begin{itemize}
        \item{Weird people use emacs}
        \end{itemize}
    \end{itemize}
}
\subsection{Other Commands}
\frame
{
    \frametitle{Other Commands}
    \begin{itemize}
    \item{man}
    \item{Ctrl-C}
    \item{backtick (`)}
    \item{screen}
    \end{itemize}
}
\section{Questions}
\frame
{	
    \frametitle{Questions?}
    Questions?
}
\section{About LUG/ACM}
\frame
{	
%    \frametitle{About the UIC-LUG & UIC-ACM}
    UIC LUG
    \begin{itemize}
    \item{Meet on Thursdays at 5:00 PM}
    \item{Meetings in SEL 2260 (Next Door)}
    \end{itemize}
    UIC ACM
    \begin{itemize}
    \item{ACM Office: SEL East 2262}
    \item{Meet on select Thursdays at 6:00 PM}
    \item{Meetings in SEO 1000}
    \end{itemize}
}
\end{document}

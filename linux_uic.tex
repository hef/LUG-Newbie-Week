\documentclass[hyperref={pdfpagelabels=false}]{beamer}
\usepackage{beamerthemesplit}
\usepackage{lmodern}
\title{Using linux at UIC}
\author{hef}
\date{\today}
\begin{document}
\frame{\titlepage}
\section[outline]{}
\frame{\tableofcontents}
\section{wireless}
\subsection{about}
\frame{802.1x}
\subsection{network manager}
\frame
{
    \frametitle{network manager}
    TODO: insert screenshot
}
\frame
{
    \frametitle{conneting to the network}
    TODO: insert screenshot
    security: dynamic wep (802.1x)
}
\frame
{
    \frametitle{network settings}
    \begin{itemize}
    \item{Network name: UIC-Wireless}
    \item{Wireless Security: Dynamic WEP (802.1x)}
    \item{Authentication: Tunneled TLS}
    \item{Anonymous Identity: anonymous}
    \item{inner Authentication: PAP}
    \item{Username: ACCC NETID}
    \item{Password: ACCC PASSWORD}
    \end{itemize}
}
\frame
{
    \frametitle{Pray}
    Pray...
}
\section{turnin}
\subsection{what is turnin}
\frame
{
    \frametitle{What is turnin}
    turnin is a a unix program for turning in CS programing assignments.
}

\subsection{features}
\frame
{
    \frametitle{Features of turnin}
    \begin{itemize}
    \item{Verify that your program was turned in successfully}
    \item{turn in asignment multiple times (until the deadline)}
    \end{itemize}
}
\subsection{usage}
\frame
{
    \frametitle{checking list of assignments}
}
\frame
{
    \frametitle{turning in an assignment}
}
\frame
{
    \frametitle{ensuring that an assignment was turned in}
}
\frame
{
    \frametitle{turining in an assignment a second time}
}
\section{uic servers}
\subsection{icarus}
\subsection{bert}
\subsection{webspace}
\frame{how to use webspace}
\subsection{cs email forwarding}
\frame{how to forward cs email}
\section{getting help}
\subsection{ACCC}
\subsection{}
\end{document}

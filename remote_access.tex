\documentclass[hyperref={pdfpagelabels=false}]{beamer}
%\documentclass{article}
%\usepackage{beamerarticle}
\usepackage{beamerthemesplit}
\usepackage{lmodern}
\title{Connecting to Linux From Other Systems}
\author{UIC Linux Users Group}
\date{\today}
\begin{document}
\frame{\titlepage}
\section{ssh}
\subsection{ssh: The Secure SHell}
\frame
{
  \frametitle{ssh: The Secure SHell}
  Used to access a linux computer from anywhere
}
\subsection{Bert Example}
\frame
{
  \frametitle{Bert Example}
  hef@acm:\~\$ ssh ssennebo@bert.cs.uic.edu\\
  The authenticity of host 'bert.cs.uic.edu (131.193.40.32)' can't be established.\\
  RSA key fingerprint is 99:6a:e7:86:1f:de:19:fd:33:05:33:e8:0b:b2:72:b8.\\
  Are you sure you want to continue connecting (yes/no)? yes\\
  Warning: Permanently added 'bert.cs.uic.edu,131.193.40.32\\
  $[$ssennebo@bert$]$ \~\$
}
\subsection{How It Works}
\frame
{
  \frametitle{Objectives of Private/Public Keypair Communication}
  \begin{itemize}
  \item{secure}
  \item{ability to verify identify does not enable imitation of identity}
  \item{a recorded network sessions cannot be replayed or reproduced by either side}
  \end{itemize}
}
\subsection{Public Key Encryption}
\frame
{
    \frametitle{Public Key Encryption}
    \begin{itemize}
    \item{A keypair consists of 2 parts: a private key and a public key}
    \item{A public key is shared freely}
    \item{A private key is kept secret}
    \end{itemize}
}
\subsection{Server Fingerprint}
\frame
{
    \frametitle{The Fingerprint}
    \begin{itemize}
    \item{The fingerprint is a hash of the servers public key}
    \item{The client sends a message by encrypting data with the servers public key}
    \item{The server uses its private key to decrypt the message}
    \item{The server sends a response that the client uses to verify the identity of the server.}
    \end{itemize}
}
\subsection{SSH Config}
\frame
{
	\frametitle{.ssh/config}
	\begin{block}{.ssh/config}
	Host acm\\
	Hostname acm.cs.uic.edu\\
	User alice\\
	Port 22
	\end{block}
}
\begin{frame}[fragile]
\frametitle{Without .ssh/config}
\begin{verbatim}
ssh alice@acm.cs.uic.edu -p 22
scp -P 22 alice@acm.cs.uic.edu:/ ~/acm/
rsync -e 'ssh -p 22' alice@acm.cs.uic.edu:~/ ~/acm/
\end{verbatim}
\end{frame}
\begin{frame}[fragile]
\frametitle{With .ssh/config}
\begin{verbatim}
ssh acm
scp acm:/ ~/acm/
rsync acm:~/ ~/acm/
\end{verbatim}
\end{frame}
\section{File Transfer}
\subsection{SCP a File}
\begin{frame}[fragile]
    \frametitle{scp: secure copy}
\begin{verbatim}
scp localfile user@remotehost:~/path/to/destination/
scp user@remotehost:~/path/to/file localfile
\end{verbatim}
\end{frame}
\subsection{SCP a Directory}
\begin{frame}[fragile]
    \frametitle{scp: secure copy}
\begin{verbatim}
scp -r ~/localdir user@remotehost:~/remotedir/
scp -r user@remotehost:~/directory ~/localdir/
\end{verbatim}
\end{frame}
\subsection{Rsync}
\frame
{
    \frametitle{rsync}
    \begin{itemize}
    \item{Has prettier output, and can do updated file data only (syncing)}
    \item{rsync -auP localfile user@remotehost:/path/to/destination}
    \end{itemize}
}
\section{Passwordless Logins}
\subsection{SSH Keys}
\frame
{
    \frametitle{SSH Keys}
    \begin{itemize}
    \item{SSH keys can be used as a secure alternative to password based logins}
    \item{useful for ssh base version control systems like git, mercurial, and subversion}
    \end{itemize}
}
\subsection{How a Keypair Works}
\frame
{
    \frametitle{How a Keypair Works}
    \begin{itemize}
    \item{similar in concept to fingerprint identification}
    \item{the server has your public key}
    \item{the server send a challenge message by encrypting data against your public key}
    \item{you use your private key to decrypt the message and prove your identity}
    \end{itemize}
}
\subsection{Creating a SSH Keypair}
\frame
{
    \frametitle{Creating a SSH Keypair}
    \begin{itemize}
    \item{ssh-keygen}
    \item{save file in default location (\~/.ssh/id\_rsa)}
    \item{enter a passphrase}
    \item{confirm passphrase}
    \end{itemize}
}
\subsection{Copy Public Key}
\frame
{
    \frametitle{Copy Public Key}
    \begin{itemize}
    \item{copy public key (~/.ssh/id\_rsa.pub) to host computers in ~/.ssh/authorized\_keys2}
    \item{ssh-copy-id can automate this task}
    \end{itemize}
}
\subsection{ssh-agent}
\frame
{
    \frametitle{ssh-agent}
    an ssh-agent is a program that can keep your decrypted keys in memory, so that you only need to enter your passphrase once per session
}
\frame
{
    \frametitle{ssh-agent}
    Most window managers come with an ssh-agent that will ask for your passphrase the first time you use a key, including gnome(ubuntu) and kde.  The command line program `ssh-agent` can be used in instances where an existing ssh-agent is not available.
}
\section{Persistance}
\subsection{The Problem}
\frame
{
    \frametitle{The problem}
    With ssh, or any shell application, any program you launch becomes a child of that shell.  When you exit the shell, the program also exits.

}
\subsection{Disown the Process}
\frame
{
  \frametitle{Disown the Process}
  \begin{itemize}
  \item{press ctrl + z to suspend the process}
  \item{run `bg` to background suspended processes}
  \item{run `disown -h` to disown background processes}
  \end{itemize}
}
\frame
{
    \frametitle{Disadvantages}
    \begin{itemize}
    \item{cannot reconnect to the process}
    \end{itemize}
}
\subsection{Screen}
\frame
{
  \frametitle{Screen}
  screen is a Terminal Multiplexor.
  This lets us create screen sessions, and disconnect and reconnect to them freely.
}
\subsection{Using Screen}
\frame
{
  \frametitle{Using Screen}
  \begin{itemize}
  \item{screen -R irc}
  \item{run irssi}
  \item{press ctrl + a,d}
  \item{connect from somewhere else (or not)}
  \item{run screen -R irc}
  \end{itemize}
}

\section{Connecting From Other Platforms}
\subsection{Putty}
\frame
{
    \frametitle{Putty}
    \begin{itemize}
    \item{windows based ssh client}
    \item{support ssh-agent like abilities with paegent}
    \end{itemize}
}
\subsection{WinSCP}
\frame
{
    \frametitle{WinSCP}
    \begin{itemize}
    \item{windows based file transfer client}
    \end{itemize}
}
\subsection{Fugu}
\frame
{
    \frametitle{Fugu}
    \begin{itemize}
    \item{OS X based Secure File Transfer client}
    \end{itemize}
}
\end{document}

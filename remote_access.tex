\documentclass[hyperref={pdfpagelabels=false}]{beamer}
\usepackage{beamerthemesplit}
\usepackage{lmodern}
\title{Connecting to linux from other systems}
\author{hef}
\date{\today}
\begin{document}
\frame{\titlepage}
\section[outline]{}
\frame[shrink]{\tableofcontents}
\section{ssh}
\subsection{ssh: The Secure SHell}
\frame
{
  \frametitle{ssh: the secure shell}
  Used to access your linux computer from anywhere
}
\subsection{bert example}
\frame
{
  \frametitle{bert example}
  hef@acm:\~\$ ssh ssennebo@bert.cs.uic.edu\\
  The authenticity of host 'bert.cs.uic.edu (131.193.40.32)' can't be established.\\
  RSA key fingerprint is 99:6a:e7:86:1f:de:19:fd:33:05:33:e8:0b:b2:72:b8.\\
  Are you sure you want to continue connecting (yes/no)? yes\\
  Warning: Permanently added 'bert.cs.uic.edu,131.193.40.32\\
  $[$ssennebo@bert$]$ \~\$
}
\subsection{how it works}
\frame
{
  \frametitle{objectives of private/public keypair communication}
  \begin{itemize}
  \item{secure}
  \item{ability to verify identify does not enable imitation of identity}
  \item{a recorded network sessions cannot be replayed or reproduced by either side}
  \end{itemize}
}
\subsection{public key encryption}
\frame
{
    \frametitle{public key encryption}
    \begin{itemize}
    \item{A keypair consists of 2 parts: a private key and a public key}
    \item{A public key is shared freely}
    \item{A private key is kept secret}
    \end{itemize}
}
\subsection{Server Fingerprint}
\frame
{
    \frametitle{The FingerPrint}
    \begin{itemize}
    \item{The fingerprint is a public key}
    \item{The client sends a message by encrypting data with the servers public key}
    \item{The server uses its private key to decrypt the message}
    \item{The server sends a response that the client uses to verify the identity of the server.}
    \end{itemize}
}
\section{persistance}
\subsection{The problem}
\frame
{
    \frametitle{The problem}
    With ssh, or any shell applicaiton, any program you launch becomes a child of that shell.  when you exit the shell, the program also exits.

}
\subsection{disown the process}
\frame
{
  \frametitle{disown the process}
  \begin{itemize}
  \item{press ctrl + z to suspend the process}
  \item{run `bg` to background suspened processes}
  \item{run `disown -h` to disown background processes}
  \end{itemize}
}
\frame
{
    \frametitle{disadvantages}
    \begin{itemize}
    \item{cannot reconnect to the process}
    \end{itemize}
}
\subsection{option 2: screen}
\frame
{
  \frametitle{screen}
  screen is a Terminal Multiplexor.\\
  This lets us create screen sessions, and disconnect and reconnect to the freely.
}
\subsection{using screen}
\frame
{
  \frametitle{using screen}
  \begin{itemize}
  \item{screen -R irc}
  \item{run irssi}
  \item{press ctrl + a,d}
  \item{connect from somewhere else (or not)}
  \item{run screen -R irc}
  \end{itemize}
}
\section{passwordless logins}
\subsection{ssh keys}
\subsection{how a keypair works}
\frame
{
    \frametitle{how a keypair works}
    \begin{itemize}
    \item{similar in concept to fingerprint identification}
    \item{the server has your public key}
    \item{the server send a challange message by encrypting data against your public key}
    \item{you use your private key to decrypt the message and prove your identity}
    \end{itemize}
}
\subsection{creating a ssh keypair}
\frame
{
    \frametitle{creating a ssh keypair}
    \begin{itemize}
    \item{ssh-keygen}
    \item{save file in defalt location (~/.ssh/id\_rsa)}
    \item{enter a passphrase}
    \item{confirm passphrase}
    \end{itemize}
}
\subsection{copy public key}
\frame
{
    \frametitle{copy public key}
    \begin{itemize}
    \item{copy public key (~/.ssh/id\_rsa.pub) to host computers in ~/.ssh/authorized\_keys2}
    \item{ssh-copy-id can automate this task}
    \end{itemize}
}
\subsection{ssh-agent}
\frame
{
    \frametitle{ssh-agent}
    an ssh-agent is a program that can keep your decrypted keys in memory, so that you only need to enter your passphrase once per session
}
\frame
{
    \frametitle{ssh-agent}
    Most window manager come with an ssh-agent that will ask for your passphrase the first time you use a key, including gnome(ubuntu) and kde.  The commandline program `ssh-agent` can be used in instances where an existing ssh-agent is not available.
}
\section{connecting from other platforms}
\subsection{putty}
\subsection{winscp}
\subsection{fugu}
\section{ssh keys}
\end{document}

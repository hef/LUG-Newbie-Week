\documentclass[hyperref={pdfpagelabels=false}]{beamer}
\usepackage{beamerthemesplit}
\usepackage{lmodern}
\title{Connecting to linux from other systems}
\author{hef}
\date{\today}
\begin{document}
\frame{\titlepage}
\section[outline]{}
\frame{\tableofcontents}
\section{ssh}
\subsection{ssh: the {\bf s}ecure {\bf sh}ell}
\frame
{
  \frametitle{ssh: the secure shell}
  Used to access your linux computer from anywhere
}
\subsection{bert example}
\frame
{
  \frametitle{bert example}
  hef@acm:\~\$ ssh ssennebo@bert.cs.uic.edu\\
  The authenticity of host 'bert.cs.uic.edu (131.193.40.32)' can't be established.\\
  RSA key fingerprint is 99:6a:e7:86:1f:de:19:fd:33:05:33:e8:0b:b2:72:b8.\\
  Are you sure you want to continue connecting (yes/no)? yes\\
  Warning: Permanently added 'bert.cs.uic.edu,131.193.40.32\\
  $[$ssennebo@bert$]$ \~\$
}
\subsection{how it works}
\frame
{
  \frametitle{objectives of private/public keypair communication}
  \begin{itemize}
  \item{secure}
  \item{ability to verify identify does not enable imitation of identity}
  \item{a recorded network sessions cannot be replayed or reproduced by either side}
  \end{itemize}
}
\section{persistance}
\subsection{disown the process}
\frame
{
  \frametitle{disown the process}
  \begin{itemize}
  \item{press ctrl + z}
  \item{run `bg`}
  \item{run `disown -h`}
  \end{itemize}
}
\subsection{screen}
\frame
{
  \frametitle{screen}
  screen is a Terminal Multiplexor.\\
  This lets us create screen sessions, and disconnect and reconnect to the freely.
}
\frame
{
  \frametitle{using screen}
  \begin{itemize}
  \item{screen -R irc}
  \item{run irssi}
  \item{press ctrl + a,d}
  \item{connect from somewhere else (or not)}
  \item{run screen -R irc}
  \end{itemize}
}

\section{connecting from windows}
\subsection{putty}
\subsection{winscp}
\section{connecting from OS X}
\subsection{ssh}
\subsection{fugu}
\section{connecting from linux}
\subsection{ssh}
\subsection{rsync}
\section{ssh keys}
\subsection{how does a keypair work}
\subsection{creating your keypair}
\subsection{copying your public key}
\subsection{useing an ssh-agent}
\end{document}
